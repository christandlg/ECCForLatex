\documentclass[]{report}

% Title Page
\title{}
\author{}

\usepackage{tikz}
\usepackage{circuitikz}

%path to pgfcircspecial.tex
%%file containing definition of special circuitikz components
%adapted from: https://tex.stackexchange.com/questions/98453/how-can-i-add-a-component-to-circuitikz

\makeatletter

%warburg impedance
\pgfcircdeclarebipole{}{\ctikzvalof{bipoles/diode/height}}{warburgimpedance}{\ctikzvalof{bipoles/diode/height}}{\ctikzvalof{bipoles/diode/width}}{
	\pgfsetlinewidth{\pgfkeysvalueof{/tikz/circuitikz/bipoles/thickness}\pgfstartlinewidth}
	\pgfscope
	\pgftransformxshift{\pgf@circ@res@left}
	\pgfpathmoveto{\pgfpoint{0pt}{0.8\pgf@circ@res@up}}
	\pgfpathlineto{\pgfpoint{\pgf@circ@res@right+0.5\pgf@circ@res@left}{\pgf@circ@res@down}}	\pgfpathlineto{\pgfpoint{\pgf@circ@res@right-0\pgf@circ@res@left}{0.8\pgf@circ@res@up}}
	\pgfpathlineto{\pgfpoint{\pgf@circ@res@right-0.5\pgf@circ@res@left}{\pgf@circ@res@down}}	\pgfpathlineto{\pgfpoint{\pgf@circ@res@right-\pgf@circ@res@left}{0.8\pgf@circ@res@up}}

	\pgfusepath{draw}
	\endpgfscope
}

%electrochemical cell (2 electrodes)
\pgfdeclareshape{ecc2e}
{
	\savedanchor\northwest{
		\pgf@y=\pgfkeysvalueof{/tikz/circuitikz/bipoles/twoport/width}\pgf@circ@Rlen
		\pgf@y=.5\pgf@y
		\pgf@x=\pgfkeysvalueof{/tikz/circuitikz/bipoles/twoport/width}\pgf@circ@Rlen
		\pgf@x=-.5\pgf@x
	}
	\anchor{center}{
		\pgfpointorigin
	}
	\anchor{left}{%
		\northwest
		\pgf@y=0pt
	}
	\anchor{in 1}{
		\northwest
		\pgf@y=0pt
	}
	\anchor{in1}{
		\northwest
		\pgf@y=0pt
	}
	\anchor{in}{
		\northwest
		\pgf@y=0pt
	}
	\anchor{aux}{
		\northwest
		\pgf@y=0pt
	}
	\anchor{ce}{
		\northwest
		\pgf@y=0pt
	}
	\anchor{in 2}{
		\northwest
		\pgf@y=0pt
		\pgf@x=-\pgf@x
	}
	\anchor{in2}{
		\northwest
		\pgf@y=0pt
		\pgf@x=-\pgf@x
	}
	\anchor{we}{
		\northwest
		\pgf@y=0pt
		\pgf@x=-\pgf@x
	}
	\anchor{out}{
		\northwest
		\pgf@y=\pgf@y
		\pgf@x=0pt
	}
	\anchor{out1}{
		\northwest
		\pgf@y=\pgf@y
		\pgf@x=0pt
	}
	\anchor{re}{
		\northwest
		\pgf@y=\pgf@y
		\pgf@x=0pt
	}
	\anchor{re1}{
		\northwest
		\pgf@y=\pgf@y
		\pgf@x=0pt
	}
	\anchor{out2}{
		\northwest
		\pgf@y=-\pgf@y
		\pgf@x=0pt
	}
	\anchor{re2}{
		\northwest
		\pgf@y=-\pgf@y
		\pgf@x=0pt
	}
	\anchor{center}{
		\pgf@y=0pt
		\pgf@x=0pt
	}
	\anchor{east}{
		\northwest
		\pgf@y=0pt
		\pgf@x=-\pgf@x  
	}
	\anchor{west}{
		\northwest
		\pgf@y=0pt
	}
	\anchor{south}{
		\northwest
		\pgf@x=0pt
		\pgf@y=-\pgf@y
	}
	\anchor{north}{
		\northwest
		\pgf@x=0pt
	}
	\anchor{south west}{
		\northwest
		\pgf@y=-\pgf@y
	}
	\anchor{north east}{
		\northwest
		\pgf@x=-\pgf@x
	}
	\anchor{north west}{
		\northwest
	}
	\anchor{south east}{
		\northwest
		\pgf@x=-\pgf@x
		\pgf@y=-\pgf@y
	}
	\backgroundpath{
		\pgfsetcolor{\pgfkeysvalueof{/tikz/circuitikz/color}}	
		
		\pgf@circ@res@step=\ctikzvalof{tripoles/mixer/width}\pgf@circ@Rlen
		
		\pgfscope
		\pgfstartlinewidth=\pgflinewidth
		
		% draw outer circle
		\pgfsetlinewidth{\pgfkeysvalueof{/tikz/circuitikz/bipoles/thickness}\pgfstartlinewidth}
		\pgfpathcircle{\pgfpoint{0}{0}} {0.5\pgf@circ@res@step}
		\pgfusepath{draw}
		
		% draw inner stuff
		\pgfsetlinewidth{\pgfstartlinewidth}
		
		%counter electrode (in1)
		\pgfpathmoveto{\pgfpoint{-0.5\pgf@circ@res@step}{0pt}}
		\pgfpathlineto{\pgfpoint{-0.3\pgf@circ@res@step}{0pt}}
		\pgfpathmoveto{\pgfpoint{-0.3\pgf@circ@res@step}{0.25\pgf@circ@res@step}}
		\pgfpathlineto{\pgfpoint{-0.3\pgf@circ@res@step}{-0.25\pgf@circ@res@step}}
		
		%working electrode (in2)
		\pgfpathmoveto{\pgfpoint{0.5\pgf@circ@res@step}{0pt}}
		\pgfpathlineto{\pgfpoint{0.3\pgf@circ@res@step}{0pt}}
		\pgfpathmoveto{\pgfpoint{0.3\pgf@circ@res@step}{0.25\pgf@circ@res@step}}
		\pgfpathlineto{\pgfpoint{0.3\pgf@circ@res@step}{-0.25\pgf@circ@res@step}}
		
		\pgfusepath{draw}		  
		
		\endpgfscope	
	}
}


%electrochemical cell (3 electrodes)
\pgfdeclareshape{ecc3e}
{
	\savedanchor\northwest{
		\pgf@y=\pgfkeysvalueof{/tikz/circuitikz/bipoles/twoport/width}\pgf@circ@Rlen
		\pgf@y=.5\pgf@y
		\pgf@x=\pgfkeysvalueof{/tikz/circuitikz/bipoles/twoport/width}\pgf@circ@Rlen
		\pgf@x=-.5\pgf@x
	}
	\anchor{center}{
		\pgfpointorigin
	}
	\anchor{left}{%
		\northwest
		\pgf@y=0pt
	}
	\anchor{in 1}{
		\northwest
		\pgf@y=0pt
	}
	\anchor{in1}{
		\northwest
		\pgf@y=0pt
	}
	\anchor{in}{
		\northwest
		\pgf@y=0pt
	}
	\anchor{aux}{
		\northwest
		\pgf@y=0pt
	}
	\anchor{ce}{
		\northwest
		\pgf@y=0pt
	}	  
	\anchor{in 2}{
		\northwest
		\pgf@y=0pt
		\pgf@x=-\pgf@x
	}
	\anchor{in2}{
		\northwest
		\pgf@y=0pt
		\pgf@x=-\pgf@x
	}
	\anchor{we}{
		\northwest
		\pgf@y=0pt
		\pgf@x=-\pgf@x
	}
	\anchor{out}{
		\northwest
		\pgf@y=\pgf@y
		\pgf@x=0pt
	}
	\anchor{out1}{
		\northwest
		\pgf@y=\pgf@y
		\pgf@x=0pt
	}
	\anchor{re}{
		\northwest
		\pgf@y=\pgf@y
		\pgf@x=0pt
	}
	\anchor{re1}{
		\northwest
		\pgf@y=\pgf@y
		\pgf@x=0pt
	}
	\anchor{out2}{
		\northwest
		\pgf@y=-\pgf@y
		\pgf@x=0pt
	}
	\anchor{re2}{
		\northwest
		\pgf@y=-\pgf@y
		\pgf@x=0pt
	}
	\anchor{center}{
		\pgf@y=0pt
		\pgf@x=0pt
	}
	\anchor{east}{
		\northwest
		\pgf@y=0pt
		\pgf@x=-\pgf@x  
	}
	\anchor{west}{
		\northwest
		\pgf@y=0pt
	}
	\anchor{south}{
		\northwest
		\pgf@x=0pt
		\pgf@y=-\pgf@y
	}
	\anchor{north}{
		\northwest
		\pgf@x=0pt
	}
	\anchor{south west}{
		\northwest
		\pgf@y=-\pgf@y
	}
	\anchor{north east}{
		\northwest
		\pgf@x=-\pgf@x
	}
	\anchor{north west}{
		\northwest
	}
	\anchor{south east}{
		\northwest
		\pgf@x=-\pgf@x
		\pgf@y=-\pgf@y
	}
	\backgroundpath{
		\pgfsetcolor{\pgfkeysvalueof{/tikz/circuitikz/color}}	
		
		\pgf@circ@res@step=\ctikzvalof{tripoles/mixer/width}\pgf@circ@Rlen
		
		\pgfscope
		\pgfstartlinewidth=\pgflinewidth
		
		% draw outer circle
		\pgfsetlinewidth{\pgfkeysvalueof{/tikz/circuitikz/bipoles/thickness}\pgfstartlinewidth}
		\pgfpathcircle{\pgfpoint{0}{0}} {0.5\pgf@circ@res@step}
		\pgfusepath{draw}
		
		% draw inner stuff
		\pgfsetlinewidth{\pgfstartlinewidth}

		%auxiliary electrode (in1)
		\pgfpathmoveto{\pgfpoint{-0.5\pgf@circ@res@step}{0pt}}
		\pgfpathlineto{\pgfpoint{-0.3\pgf@circ@res@step}{0pt}}
		\pgfpathmoveto{\pgfpoint{-0.3\pgf@circ@res@step}{0.25\pgf@circ@res@step}}
		\pgfpathlineto{\pgfpoint{-0.3\pgf@circ@res@step}{-0.25\pgf@circ@res@step}}
		
		%working electrode (in2)
		\pgfpathmoveto{\pgfpoint{0.5\pgf@circ@res@step}{0pt}}
		\pgfpathlineto{\pgfpoint{0.3\pgf@circ@res@step}{0pt}}
		\pgfpathmoveto{\pgfpoint{0.3\pgf@circ@res@step}{0.25\pgf@circ@res@step}}
		\pgfpathlineto{\pgfpoint{0.3\pgf@circ@res@step}{-0.25\pgf@circ@res@step}}
		
		%reference electrode (out)
		\pgfpathmoveto{\pgfpoint{0pt}{0.5\pgf@circ@res@step}}
		\pgfpathlineto{\pgfpoint{0pt}{0.2\pgf@circ@res@step}}
		\pgfpathlineto{\pgfpoint{0.2\pgf@circ@res@step}{0pt}}
		\pgfpathlineto{\pgfpoint{0.2\pgf@circ@res@step}{0.1\pgf@circ@res@step}}
		\pgfpathmoveto{\pgfpoint{0.2\pgf@circ@res@step}{0pt}}
		\pgfpathlineto{\pgfpoint{0.1\pgf@circ@res@step}{0pt}}
		
		\pgfusepath{draw}		  
		
		\endpgfscope	
	}
}
		
%electrochemical cell (4 electrodes)
\pgfdeclareshape{ecc4e}
{
	\savedanchor\northwest{
		\pgf@y=\pgfkeysvalueof{/tikz/circuitikz/bipoles/twoport/width}\pgf@circ@Rlen
		\pgf@y=.5\pgf@y
		\pgf@x=\pgfkeysvalueof{/tikz/circuitikz/bipoles/twoport/width}\pgf@circ@Rlen
		\pgf@x=-.5\pgf@x
	}
	\anchor{center}{
		\pgfpointorigin
	}
	\anchor{left}{%
		\northwest
		\pgf@y=0pt
	}
	\anchor{in 1}{
		\northwest
		\pgf@y=0pt
	}
	\anchor{in1}{
		\northwest
		\pgf@y=0pt
	}
	\anchor{in}{
		\northwest
		\pgf@y=0pt
	}
	\anchor{aux}{
		\northwest
		\pgf@y=0pt
	}
	\anchor{ce}{
		\northwest
		\pgf@y=0pt
	}	  
	\anchor{in 2}{
		\northwest
		\pgf@y=0pt
		\pgf@x=-\pgf@x
	}
	\anchor{in2}{
		\northwest
		\pgf@y=0pt
		\pgf@x=-\pgf@x
	}
	\anchor{we}{
		\northwest
		\pgf@y=0pt
		\pgf@x=-\pgf@x
	}
	\anchor{out}{
		\northwest
		\pgf@y=\pgf@y
		\pgf@x=0pt
	}
	\anchor{out1}{
		\northwest
		\pgf@y=\pgf@y
		\pgf@x=0pt
	}
	\anchor{re}{
		\northwest
		\pgf@y=\pgf@y
		\pgf@x=0pt
	}
	\anchor{re1}{
		\northwest
		\pgf@y=\pgf@y
		\pgf@x=0pt
	}
	\anchor{out2}{
		\northwest
		\pgf@y=-\pgf@y
		\pgf@x=0pt
	}
	\anchor{re2}{
		\northwest
		\pgf@y=-\pgf@y
		\pgf@x=0pt
	}
	\anchor{center}{
		\pgf@y=0pt
		\pgf@x=0pt
	}
	\anchor{east}{
		\northwest
		\pgf@y=0pt
		\pgf@x=-\pgf@x  
	}
	\anchor{west}{
		\northwest
		\pgf@y=0pt
	}
	\anchor{south}{
		\northwest
		\pgf@x=0pt
		\pgf@y=-\pgf@y
	}
	\anchor{north}{
		\northwest
		\pgf@x=0pt
	}
	\anchor{south west}{
		\northwest
		\pgf@y=-\pgf@y
	}
	\anchor{north east}{
		\northwest
		\pgf@x=-\pgf@x
	}
	\anchor{north west}{
		\northwest
	}
	\anchor{south east}{
		\northwest
		\pgf@x=-\pgf@x
		\pgf@y=-\pgf@y
	}
	\backgroundpath{
		\pgfsetcolor{\pgfkeysvalueof{/tikz/circuitikz/color}}	
		
		\pgf@circ@res@step=\ctikzvalof{tripoles/mixer/width}\pgf@circ@Rlen
		
		\pgfscope
		\pgfstartlinewidth=\pgflinewidth
		
		% draw outer circle
		\pgfsetlinewidth{\pgfkeysvalueof{/tikz/circuitikz/bipoles/thickness}\pgfstartlinewidth}
		\pgfpathcircle{\pgfpoint{0}{0}} {0.5\pgf@circ@res@step}
		\pgfusepath{draw}
		
		% draw inner stuff
		\pgfsetlinewidth{\pgfstartlinewidth}
		
		%auxiliary electrode (in1)
		\pgfpathmoveto{\pgfpoint{-0.5\pgf@circ@res@step}{0pt}}
		\pgfpathlineto{\pgfpoint{-0.3\pgf@circ@res@step}{0pt}}
		\pgfpathmoveto{\pgfpoint{-0.3\pgf@circ@res@step}{0.25\pgf@circ@res@step}}
		\pgfpathlineto{\pgfpoint{-0.3\pgf@circ@res@step}{-0.25\pgf@circ@res@step}}
		
		%working electrode (in2)
		\pgfpathmoveto{\pgfpoint{0.5\pgf@circ@res@step}{0pt}}
		\pgfpathlineto{\pgfpoint{0.3\pgf@circ@res@step}{0pt}}
		\pgfpathmoveto{\pgfpoint{0.3\pgf@circ@res@step}{0.25\pgf@circ@res@step}}
		\pgfpathlineto{\pgfpoint{0.3\pgf@circ@res@step}{-0.25\pgf@circ@res@step}}
		
		%reference electrode (out1)
		\pgfpathmoveto{\pgfpoint{0pt}{0.5\pgf@circ@res@step}}
		\pgfpathlineto{\pgfpoint{0pt}{0.2\pgf@circ@res@step}}
		\pgfpathlineto{\pgfpoint{0.2\pgf@circ@res@step}{0pt}}
		\pgfpathlineto{\pgfpoint{0.2\pgf@circ@res@step}{0.1\pgf@circ@res@step}}
		\pgfpathmoveto{\pgfpoint{0.2\pgf@circ@res@step}{0pt}}
		\pgfpathlineto{\pgfpoint{0.1\pgf@circ@res@step}{0pt}}
		
		%reference electrode (out2)
		\pgfpathmoveto{\pgfpoint{0pt}{-0.5\pgf@circ@res@step}}
		\pgfpathlineto{\pgfpoint{0pt}{-0.2\pgf@circ@res@step}}
		\pgfpathlineto{\pgfpoint{-0.2\pgf@circ@res@step}{0pt}}
		\pgfpathlineto{\pgfpoint{-0.2\pgf@circ@res@step}{-0.1\pgf@circ@res@step}}
		\pgfpathmoveto{\pgfpoint{-0.2\pgf@circ@res@step}{0pt}}
		\pgfpathlineto{\pgfpoint{-0.1\pgf@circ@res@step}{0pt}}
		
		\pgfusepath{draw}		  
		
		\endpgfscope	
	}
}
		
		
		

\tikzset{
	WI/.style={Warburg Impedance},
	Warburg Impedance/.style={\circuitikzbasekey, /tikz/to path=\pgf@circ@warburgimpedance@path},
}
\def\pgf@circ@warburgimpedance@path#1{\pgf@circ@bipole@path{warburgimpedance}{#1}}
\makeatother




%%adapted from: https://tex.stackexchange.com/questions/19204/circuitikz-create-new-component
%\newcommand{\warburgimpedance}[1]
%{
%	\coordinate (a) at (#1.nw) ;
%	\coordinate (b) at ($(#1.sw)!0.5!(#1.s)$);
%	\coordinate (c) at (#1.n);
%	\coordinate (d) at ($(#1.se)!0.5!(#1.s)$);
%	\coordinate (e) at (#1.ne);
%	\draw[thick] (a)--(b)--(c)--(d)--(e);
%}

\begin{document}	
	
\begin{figure}
	\begin{center}
		\begin{circuitikz}[scale=0.75, transform shape, european]
			\draw
			
			%electrochemical cell
			(0,0) node [ecc2e] (ecc1) {}
			;
		\end{circuitikz}
		\caption[ECC - 2E]{Electrochemical Cell with 2 Electrodes.}
	\end{center}
\end{figure}

\begin{figure}
\begin{center}
	\begin{circuitikz}[scale=0.75, transform shape, european]
		\draw
		
		%electrochemical cell
		(0,0) node [ecc3e] (ecc1) {}
		;
	\end{circuitikz}
	\caption[ECC - 3E]{Electrochemical Cell with 3 Electrodes.}
\end{center}
\end{figure}

\begin{figure}
\begin{center}
	\begin{circuitikz}[scale=0.75, transform shape, european]
		\draw
		
		%electrochemical cell
		(0,0) node [ecc4e] (ecc1) {}
		;
	\end{circuitikz}
	\caption[ECC - 4E]{Electrochemical Cell with 4 Electrodes.}
\end{center}
\end{figure}



\begin{figure}
	\begin{center}
		\begin{circuitikz}[scale=0.75, transform shape, european]
			\draw
			
			%electrochemical cell
			(-5,0) node [ecc2e] (ecc1) {}
			(ecc1.ce)
			node [below left] {CE}
			(ecc1.we)
			node [below right] {WE}
			
			
			(0,0) node [ecc3e] (ecc2) {}
			(ecc2.ce)
			node [below left] {AUX}
			(ecc2.we)
			node [below right] {WE}
			(ecc2.re)
			node [above] {RE}			
			
			(5,0) node [ecc4e] (ecc3) {}
			(ecc3.ce)
			node [below left] {AUX}
			(ecc3.we)
			node [below right] {WE}
			(ecc3.re1)
			node [above] {RE1}
			(ecc3.re2)
			node [below] {RE2}
			;
		\end{circuitikz}
		\caption[ECC - Anchors]{Symbol Anchors. }
	\end{center}
\end{figure}
	

\begin{figure}
	\begin{center}
		\begin{circuitikz}[scale=0.75, transform shape, european]
			\draw
			
			%electrochemical cell - origin
			(0,0) node [ecc2e] (ecc1) {}
			
			%cell potential arrow
			($(ecc1.we)+(0,0.5)$)
			to [open, v>=$V_{cell}$] ($(ecc1.ce)+(0,0.5)$)  
			
			(ecc1.ce)
			node [below left] {RE}
			to [short] ($(ecc1.ce)+(-1,0)$)
			to [short] ($(ecc1.ce)+(-1,-4.5)$)
			node [rground] (ground_left) {}
			
			%left hand side
			($(ecc1)+(3,-0.5)$) node [op amp, yscale=-1] (opamp) {}
			(opamp) node {OA1}
			
			(opamp.out)
			to [short, *-o] ($(opamp.out)+(1,0)$)
			node [right] {$V_{out}$}
			
			(opamp.out)
			to [R, l=$R_1$] ($(opamp.out)+(0,-2)$)
			to [R, l=$R_2$] ($(opamp.out)+(0,-4)$)
			node [rground] {}		
			
			(opamp.-)
			to [short] ($(opamp.-)+(0,-1.5)$)
			to [short, -*] ($(opamp.out)+(0,-2)$)
			
			(opamp.+)
			to [short] (ecc1.we)
			node [below right] {WE}
			
			;
		\end{circuitikz}
		\caption[Electrometer amplifier for potentiometric measurements]{Electrometer amplifier circuit for potentiometric measurements using a non-inverting operational amplifier. }
	\end{center}
\end{figure}

\begin{figure}
	\begin{center}
		\begin{circuitikz}[scale=0.75, transform shape, european]
			\draw
			
			%electrochemical cell - origin
			(0,0) node [ecc2e, rotate=90] (ecc1) {}
			
			%cell potential arrow
			($(ecc1.we)+(-0.5,0)$)
			to [open, v>=$V_{cell}$] ($(ecc1.ce)+(-0.5,0)$)  
			
			(ecc1.ce)
			node [below left] {RE}
			
			(ecc1.we)
			node [above left] {WE}
			
			%input stage, upper operation amplifier
			($(ecc1)+(3,3)$) node [op amp, yscale=-1] (opamp1) {}
			(opamp1) node {OA1}
			
			%input stage, lower operation amplifier
			($(ecc1)+(3,-3)$) node [op amp] (opamp2) {}
			(opamp2) node {OA2}
			
			(ecc1.we)
			|- (opamp1.+)
			
			(ecc1.ce)
			|- (opamp2.+)
			to [short, *-] ($(opamp2.+)+(0,-0.5)$)
			node [rground] {}
			
			%R3
			(opamp1.-)
			to [short] ($(opamp1.-)+(0,-1.5)$)
			to [R, l=$R_3$] ($(opamp2.-)+(0,1.5)$)
			to [short] (opamp2.-)
			
			%R1
			(opamp1.out)
			to [short] ($(opamp1.out)+(0,-2)$)
			to [R, l=$R_1$, *-*] ($(opamp1.-)+(0,-1.5)$)		
			
			%R2
			(opamp2.out)
			to [short] ($(opamp2.out)+(0,2)$)
			to [R, l=$R_2$, *-*] ($(opamp2.-)+(0,1.5)$)
			
			%diff amplifier stage
			($(ecc1)+(8,0)$)
			node [op amp] (opamp3) {}
			(opamp3) node {OA3}
			
			($(opamp1.out)+(0,-2)$)
			to [short] ($(opamp1.out)+(0,-2.5)$)
			to [R, l=$R_4$] (opamp3.-)
			
			($(opamp2.out)+(0,2)$)
			to [short] ($(opamp2.out)+(0,2.5)$)
			to [R, l=$R_6$] (opamp3.+)
			
			(opamp3.+)
			to [R, l=$R_7$, *-] ($(opamp3.+)+(0,-2)$)
			node [rground] {}
			
			(opamp3.out)
			to [short, *-] ($(opamp3.out)+(0,2)$)
			to [R, l=$R_5$] ($(opamp3.-)+(0,1.5)$)
			to [short, -*] (opamp3.-)
			
			(opamp3.out)
			to [short, -o] ($(opamp3.out)+(1,0)$)
			node [right] {$V_{out}$}
			
			;
		\end{circuitikz}
		\caption[Instrumentation amplifier for potentiometric measurements]{Electrometer amplifier circuit for potentiometric measurements, implemented using an instumentation amplifier.}
	\end{center}
\end{figure}

\begin{figure}
	\begin{center}
		\begin{circuitikz}[scale=0.75, transform shape, european]
			\draw
			
			%electrochemical cell - origin
			(0,0) node [ecc3e] (ecc1) {}
			
			%potentiostat stage
			($(ecc1)+(-3,0)$) node [op amp] (opamp_ps1) {}
			(opamp_ps1) node {OA1}			
			
			($(opamp_ps1.+)+(-1,0)$) node [left] {$V_{set}$}
			to [short, o-] (opamp_ps1.+)
			
			(opamp_ps1.-)
			|- ($(opamp_ps1.-)+(0,1)$)
			-| (ecc1.re)
			node [above right] {RE}
			
			(opamp_ps1.out)
			to [short] (ecc1.aux) 
			node [below left] {AUX}
			
			%current follower stage
			($(ecc1)+(3,-0.5)$) node [op amp] (opamp_cf1) {}
			(opamp_cf1) node {OA2}
			
			(opamp_cf1.-)
			to [short, i_<=$I_{cell}$] (ecc1.we) 
			node [below right] {WE}
			
			(opamp_cf1.+) 
			to [short] ($(opamp_cf1.+)+(0,-1)$) 
			node [rground] {}
			
			(opamp_cf1.-)  
			to [short, *-] ($(opamp_cf1.-)+(0,1)$) 
			to [vR, l=$R_1$	] ($(opamp_cf1.out)+(0,1.5)$)
			to [short, -*] (opamp_cf1.out)		
			
			(opamp_cf1.out)
			to [short, -o] ($(opamp_cf1.out)+(1,0)$)
			node [right] {$V_{out}$}
			
			;
		\end{circuitikz}
		\caption[Potentiostat and transimpedance amplifier]{Potentiostat and transimpedance amplifier circuit for voltammetric measurements. }
	\end{center}
\end{figure}

\begin{figure}
	\begin{center}
		\begin{circuitikz}[scale=0.75, transform shape, european]
			\draw
			%solution resistace
			(0,0) node {}
			%			to [short, o-] (1,0)
			to [R, l=$R_s$, o-*] (2,0)
			
			%parallel capacitor
			(2,0)
			to [short] (2,1)
			to [C, l=$C_{d}$] (6,1)
			to [short] (6,0)
			
			%
			(2,0)
			to [short] (2,-1)
			to [R, l=$R_{p}$] (4,-1)
			to [short] (4.5, -1)
			%			node {W}
			to [WI, l=$W$] (5.5, -1)
			
			
			(5.5, -1)
			to [short] (6,-1)
			to [short, -*] (6,0)
			to [short, -o] (7,0)
			node {}
			
			;
		\end{circuitikz}
		\caption[Randles-Ershler equivalent electrical circuit]{Randles-Ershler equivalent electrical circuit model for electrochemical interfaces. \textit{W} is the \textit{Warburg Diffusion Element}.}
	\end{center}
\end{figure}
	
\end{document}      
