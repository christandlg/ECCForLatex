%%file containing definition of special circuitikz components
%adapted from: https://tex.stackexchange.com/questions/98453/how-can-i-add-a-component-to-circuitikz

\makeatletter

%warburg impedance
\pgfcircdeclarebipole{}{\ctikzvalof{bipoles/diode/height}}{warburgimpedance}{\ctikzvalof{bipoles/diode/height}}{\ctikzvalof{bipoles/diode/width}}{
	\pgfsetlinewidth{\pgfkeysvalueof{/tikz/circuitikz/bipoles/thickness}\pgfstartlinewidth}
	\pgfscope
	\pgftransformxshift{\pgf@circ@res@left}
	\pgfpathmoveto{\pgfpoint{0pt}{0.8\pgf@circ@res@up}}
	\pgfpathlineto{\pgfpoint{\pgf@circ@res@right+0.5\pgf@circ@res@left}{\pgf@circ@res@down}}	\pgfpathlineto{\pgfpoint{\pgf@circ@res@right-0\pgf@circ@res@left}{0.8\pgf@circ@res@up}}
	\pgfpathlineto{\pgfpoint{\pgf@circ@res@right-0.5\pgf@circ@res@left}{\pgf@circ@res@down}}	\pgfpathlineto{\pgfpoint{\pgf@circ@res@right-\pgf@circ@res@left}{0.8\pgf@circ@res@up}}

	\pgfusepath{draw}
	\endpgfscope
}

%electrochemical cell (2 electrodes)
\pgfdeclareshape{ecc2e}
{
	\savedanchor\northwest{
		\pgf@y=\pgfkeysvalueof{/tikz/circuitikz/bipoles/twoport/width}\pgf@circ@Rlen
		\pgf@y=.5\pgf@y
		\pgf@x=\pgfkeysvalueof{/tikz/circuitikz/bipoles/twoport/width}\pgf@circ@Rlen
		\pgf@x=-.5\pgf@x
	}
	\anchor{center}{
		\pgfpointorigin
	}
	\anchor{left}{%
		\northwest
		\pgf@y=0pt
	}
	\anchor{in 1}{
		\northwest
		\pgf@y=0pt
	}
	\anchor{in1}{
		\northwest
		\pgf@y=0pt
	}
	\anchor{in}{
		\northwest
		\pgf@y=0pt
	}
	\anchor{aux}{
		\northwest
		\pgf@y=0pt
	}
	\anchor{ce}{
		\northwest
		\pgf@y=0pt
	}
	\anchor{in 2}{
		\northwest
		\pgf@y=0pt
		\pgf@x=-\pgf@x
	}
	\anchor{in2}{
		\northwest
		\pgf@y=0pt
		\pgf@x=-\pgf@x
	}
	\anchor{we}{
		\northwest
		\pgf@y=0pt
		\pgf@x=-\pgf@x
	}
	\anchor{out}{
		\northwest
		\pgf@y=\pgf@y
		\pgf@x=0pt
	}
	\anchor{out1}{
		\northwest
		\pgf@y=\pgf@y
		\pgf@x=0pt
	}
	\anchor{re}{
		\northwest
		\pgf@y=\pgf@y
		\pgf@x=0pt
	}
	\anchor{re1}{
		\northwest
		\pgf@y=\pgf@y
		\pgf@x=0pt
	}
	\anchor{out2}{
		\northwest
		\pgf@y=-\pgf@y
		\pgf@x=0pt
	}
	\anchor{re2}{
		\northwest
		\pgf@y=-\pgf@y
		\pgf@x=0pt
	}
	\anchor{center}{
		\pgf@y=0pt
		\pgf@x=0pt
	}
	\anchor{east}{
		\northwest
		\pgf@y=0pt
		\pgf@x=-\pgf@x  
	}
	\anchor{west}{
		\northwest
		\pgf@y=0pt
	}
	\anchor{south}{
		\northwest
		\pgf@x=0pt
		\pgf@y=-\pgf@y
	}
	\anchor{north}{
		\northwest
		\pgf@x=0pt
	}
	\anchor{south west}{
		\northwest
		\pgf@y=-\pgf@y
	}
	\anchor{north east}{
		\northwest
		\pgf@x=-\pgf@x
	}
	\anchor{north west}{
		\northwest
	}
	\anchor{south east}{
		\northwest
		\pgf@x=-\pgf@x
		\pgf@y=-\pgf@y
	}
	\backgroundpath{
		\pgfsetcolor{\pgfkeysvalueof{/tikz/circuitikz/color}}	
		
		\pgf@circ@res@step=\ctikzvalof{tripoles/mixer/width}\pgf@circ@Rlen
		
		\pgfscope
		\pgfstartlinewidth=\pgflinewidth
		
		% draw outer circle
		\pgfsetlinewidth{\pgfkeysvalueof{/tikz/circuitikz/bipoles/thickness}\pgfstartlinewidth}
		\pgfpathcircle{\pgfpoint{0}{0}} {0.5\pgf@circ@res@step}
		\pgfusepath{draw}
		
		% draw inner stuff
		\pgfsetlinewidth{\pgfstartlinewidth}
		
		%counter electrode (in1)
		\pgfpathmoveto{\pgfpoint{-0.5\pgf@circ@res@step}{0pt}}
		\pgfpathlineto{\pgfpoint{-0.3\pgf@circ@res@step}{0pt}}
		\pgfpathmoveto{\pgfpoint{-0.3\pgf@circ@res@step}{0.25\pgf@circ@res@step}}
		\pgfpathlineto{\pgfpoint{-0.3\pgf@circ@res@step}{-0.25\pgf@circ@res@step}}
		
		%working electrode (in2)
		\pgfpathmoveto{\pgfpoint{0.5\pgf@circ@res@step}{0pt}}
		\pgfpathlineto{\pgfpoint{0.3\pgf@circ@res@step}{0pt}}
		\pgfpathmoveto{\pgfpoint{0.3\pgf@circ@res@step}{0.25\pgf@circ@res@step}}
		\pgfpathlineto{\pgfpoint{0.3\pgf@circ@res@step}{-0.25\pgf@circ@res@step}}
		
		\pgfusepath{draw}		  
		
		\endpgfscope	
	}
}


%electrochemical cell (3 electrodes)
\pgfdeclareshape{ecc3e}
{
	\savedanchor\northwest{
		\pgf@y=\pgfkeysvalueof{/tikz/circuitikz/bipoles/twoport/width}\pgf@circ@Rlen
		\pgf@y=.5\pgf@y
		\pgf@x=\pgfkeysvalueof{/tikz/circuitikz/bipoles/twoport/width}\pgf@circ@Rlen
		\pgf@x=-.5\pgf@x
	}
	\anchor{center}{
		\pgfpointorigin
	}
	\anchor{left}{%
		\northwest
		\pgf@y=0pt
	}
	\anchor{in 1}{
		\northwest
		\pgf@y=0pt
	}
	\anchor{in1}{
		\northwest
		\pgf@y=0pt
	}
	\anchor{in}{
		\northwest
		\pgf@y=0pt
	}
	\anchor{aux}{
		\northwest
		\pgf@y=0pt
	}
	\anchor{ce}{
		\northwest
		\pgf@y=0pt
	}	  
	\anchor{in 2}{
		\northwest
		\pgf@y=0pt
		\pgf@x=-\pgf@x
	}
	\anchor{in2}{
		\northwest
		\pgf@y=0pt
		\pgf@x=-\pgf@x
	}
	\anchor{we}{
		\northwest
		\pgf@y=0pt
		\pgf@x=-\pgf@x
	}
	\anchor{out}{
		\northwest
		\pgf@y=\pgf@y
		\pgf@x=0pt
	}
	\anchor{out1}{
		\northwest
		\pgf@y=\pgf@y
		\pgf@x=0pt
	}
	\anchor{re}{
		\northwest
		\pgf@y=\pgf@y
		\pgf@x=0pt
	}
	\anchor{re1}{
		\northwest
		\pgf@y=\pgf@y
		\pgf@x=0pt
	}
	\anchor{out2}{
		\northwest
		\pgf@y=-\pgf@y
		\pgf@x=0pt
	}
	\anchor{re2}{
		\northwest
		\pgf@y=-\pgf@y
		\pgf@x=0pt
	}
	\anchor{center}{
		\pgf@y=0pt
		\pgf@x=0pt
	}
	\anchor{east}{
		\northwest
		\pgf@y=0pt
		\pgf@x=-\pgf@x  
	}
	\anchor{west}{
		\northwest
		\pgf@y=0pt
	}
	\anchor{south}{
		\northwest
		\pgf@x=0pt
		\pgf@y=-\pgf@y
	}
	\anchor{north}{
		\northwest
		\pgf@x=0pt
	}
	\anchor{south west}{
		\northwest
		\pgf@y=-\pgf@y
	}
	\anchor{north east}{
		\northwest
		\pgf@x=-\pgf@x
	}
	\anchor{north west}{
		\northwest
	}
	\anchor{south east}{
		\northwest
		\pgf@x=-\pgf@x
		\pgf@y=-\pgf@y
	}
	\backgroundpath{
		\pgfsetcolor{\pgfkeysvalueof{/tikz/circuitikz/color}}	
		
		\pgf@circ@res@step=\ctikzvalof{tripoles/mixer/width}\pgf@circ@Rlen
		
		\pgfscope
		\pgfstartlinewidth=\pgflinewidth
		
		% draw outer circle
		\pgfsetlinewidth{\pgfkeysvalueof{/tikz/circuitikz/bipoles/thickness}\pgfstartlinewidth}
		\pgfpathcircle{\pgfpoint{0}{0}} {0.5\pgf@circ@res@step}
		\pgfusepath{draw}
		
		% draw inner stuff
		\pgfsetlinewidth{\pgfstartlinewidth}

		%auxiliary electrode (in1)
		\pgfpathmoveto{\pgfpoint{-0.5\pgf@circ@res@step}{0pt}}
		\pgfpathlineto{\pgfpoint{-0.3\pgf@circ@res@step}{0pt}}
		\pgfpathmoveto{\pgfpoint{-0.3\pgf@circ@res@step}{0.25\pgf@circ@res@step}}
		\pgfpathlineto{\pgfpoint{-0.3\pgf@circ@res@step}{-0.25\pgf@circ@res@step}}
		
		%working electrode (in2)
		\pgfpathmoveto{\pgfpoint{0.5\pgf@circ@res@step}{0pt}}
		\pgfpathlineto{\pgfpoint{0.3\pgf@circ@res@step}{0pt}}
		\pgfpathmoveto{\pgfpoint{0.3\pgf@circ@res@step}{0.25\pgf@circ@res@step}}
		\pgfpathlineto{\pgfpoint{0.3\pgf@circ@res@step}{-0.25\pgf@circ@res@step}}
		
		%reference electrode (out)
		\pgfpathmoveto{\pgfpoint{0pt}{0.5\pgf@circ@res@step}}
		\pgfpathlineto{\pgfpoint{0pt}{0.2\pgf@circ@res@step}}
		\pgfpathlineto{\pgfpoint{0.2\pgf@circ@res@step}{0pt}}
		\pgfpathlineto{\pgfpoint{0.2\pgf@circ@res@step}{0.1\pgf@circ@res@step}}
		\pgfpathmoveto{\pgfpoint{0.2\pgf@circ@res@step}{0pt}}
		\pgfpathlineto{\pgfpoint{0.1\pgf@circ@res@step}{0pt}}
		
		\pgfusepath{draw}		  
		
		\endpgfscope	
	}
}
		
%electrochemical cell (4 electrodes)
\pgfdeclareshape{ecc4e}
{
	\savedanchor\northwest{
		\pgf@y=\pgfkeysvalueof{/tikz/circuitikz/bipoles/twoport/width}\pgf@circ@Rlen
		\pgf@y=.5\pgf@y
		\pgf@x=\pgfkeysvalueof{/tikz/circuitikz/bipoles/twoport/width}\pgf@circ@Rlen
		\pgf@x=-.5\pgf@x
	}
	\anchor{center}{
		\pgfpointorigin
	}
	\anchor{left}{%
		\northwest
		\pgf@y=0pt
	}
	\anchor{in 1}{
		\northwest
		\pgf@y=0pt
	}
	\anchor{in1}{
		\northwest
		\pgf@y=0pt
	}
	\anchor{in}{
		\northwest
		\pgf@y=0pt
	}
	\anchor{aux}{
		\northwest
		\pgf@y=0pt
	}
	\anchor{ce}{
		\northwest
		\pgf@y=0pt
	}	  
	\anchor{in 2}{
		\northwest
		\pgf@y=0pt
		\pgf@x=-\pgf@x
	}
	\anchor{in2}{
		\northwest
		\pgf@y=0pt
		\pgf@x=-\pgf@x
	}
	\anchor{we}{
		\northwest
		\pgf@y=0pt
		\pgf@x=-\pgf@x
	}
	\anchor{out}{
		\northwest
		\pgf@y=\pgf@y
		\pgf@x=0pt
	}
	\anchor{out1}{
		\northwest
		\pgf@y=\pgf@y
		\pgf@x=0pt
	}
	\anchor{re}{
		\northwest
		\pgf@y=\pgf@y
		\pgf@x=0pt
	}
	\anchor{re1}{
		\northwest
		\pgf@y=\pgf@y
		\pgf@x=0pt
	}
	\anchor{out2}{
		\northwest
		\pgf@y=-\pgf@y
		\pgf@x=0pt
	}
	\anchor{re2}{
		\northwest
		\pgf@y=-\pgf@y
		\pgf@x=0pt
	}
	\anchor{center}{
		\pgf@y=0pt
		\pgf@x=0pt
	}
	\anchor{east}{
		\northwest
		\pgf@y=0pt
		\pgf@x=-\pgf@x  
	}
	\anchor{west}{
		\northwest
		\pgf@y=0pt
	}
	\anchor{south}{
		\northwest
		\pgf@x=0pt
		\pgf@y=-\pgf@y
	}
	\anchor{north}{
		\northwest
		\pgf@x=0pt
	}
	\anchor{south west}{
		\northwest
		\pgf@y=-\pgf@y
	}
	\anchor{north east}{
		\northwest
		\pgf@x=-\pgf@x
	}
	\anchor{north west}{
		\northwest
	}
	\anchor{south east}{
		\northwest
		\pgf@x=-\pgf@x
		\pgf@y=-\pgf@y
	}
	\backgroundpath{
		\pgfsetcolor{\pgfkeysvalueof{/tikz/circuitikz/color}}	
		
		\pgf@circ@res@step=\ctikzvalof{tripoles/mixer/width}\pgf@circ@Rlen
		
		\pgfscope
		\pgfstartlinewidth=\pgflinewidth
		
		% draw outer circle
		\pgfsetlinewidth{\pgfkeysvalueof{/tikz/circuitikz/bipoles/thickness}\pgfstartlinewidth}
		\pgfpathcircle{\pgfpoint{0}{0}} {0.5\pgf@circ@res@step}
		\pgfusepath{draw}
		
		% draw inner stuff
		\pgfsetlinewidth{\pgfstartlinewidth}
		
		%auxiliary electrode (in1)
		\pgfpathmoveto{\pgfpoint{-0.5\pgf@circ@res@step}{0pt}}
		\pgfpathlineto{\pgfpoint{-0.3\pgf@circ@res@step}{0pt}}
		\pgfpathmoveto{\pgfpoint{-0.3\pgf@circ@res@step}{0.25\pgf@circ@res@step}}
		\pgfpathlineto{\pgfpoint{-0.3\pgf@circ@res@step}{-0.25\pgf@circ@res@step}}
		
		%working electrode (in2)
		\pgfpathmoveto{\pgfpoint{0.5\pgf@circ@res@step}{0pt}}
		\pgfpathlineto{\pgfpoint{0.3\pgf@circ@res@step}{0pt}}
		\pgfpathmoveto{\pgfpoint{0.3\pgf@circ@res@step}{0.25\pgf@circ@res@step}}
		\pgfpathlineto{\pgfpoint{0.3\pgf@circ@res@step}{-0.25\pgf@circ@res@step}}
		
		%reference electrode (out1)
		\pgfpathmoveto{\pgfpoint{0pt}{0.5\pgf@circ@res@step}}
		\pgfpathlineto{\pgfpoint{0pt}{0.2\pgf@circ@res@step}}
		\pgfpathlineto{\pgfpoint{0.2\pgf@circ@res@step}{0pt}}
		\pgfpathlineto{\pgfpoint{0.2\pgf@circ@res@step}{0.1\pgf@circ@res@step}}
		\pgfpathmoveto{\pgfpoint{0.2\pgf@circ@res@step}{0pt}}
		\pgfpathlineto{\pgfpoint{0.1\pgf@circ@res@step}{0pt}}
		
		%reference electrode (out2)
		\pgfpathmoveto{\pgfpoint{0pt}{-0.5\pgf@circ@res@step}}
		\pgfpathlineto{\pgfpoint{0pt}{-0.2\pgf@circ@res@step}}
		\pgfpathlineto{\pgfpoint{-0.2\pgf@circ@res@step}{0pt}}
		\pgfpathlineto{\pgfpoint{-0.2\pgf@circ@res@step}{-0.1\pgf@circ@res@step}}
		\pgfpathmoveto{\pgfpoint{-0.2\pgf@circ@res@step}{0pt}}
		\pgfpathlineto{\pgfpoint{-0.1\pgf@circ@res@step}{0pt}}
		
		\pgfusepath{draw}		  
		
		\endpgfscope	
	}
}
		
		
		

\tikzset{
	WI/.style={Warburg Impedance},
	Warburg Impedance/.style={\circuitikzbasekey, /tikz/to path=\pgf@circ@warburgimpedance@path},
}
\def\pgf@circ@warburgimpedance@path#1{\pgf@circ@bipole@path{warburgimpedance}{#1}}
\makeatother




%%adapted from: https://tex.stackexchange.com/questions/19204/circuitikz-create-new-component
%\newcommand{\warburgimpedance}[1]
%{
%	\coordinate (a) at (#1.nw) ;
%	\coordinate (b) at ($(#1.sw)!0.5!(#1.s)$);
%	\coordinate (c) at (#1.n);
%	\coordinate (d) at ($(#1.se)!0.5!(#1.s)$);
%	\coordinate (e) at (#1.ne);
%	\draw[thick] (a)--(b)--(c)--(d)--(e);
%}